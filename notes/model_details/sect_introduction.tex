% I just tried writing an introduction for myself. Let's keep this shorter as these are notes for collaborators. 
%From insects to mammals, organisms navigate complex olfactory landscapes by picking up new odors amidst fluctuating mixtures of irrelevant background odorants. Habituation to the olfactory background on the time scale of an hour is thought to help organisms discriminate the appearance of relevant odors in the olfactory landscape \cite{shen_habituation_2020}. Habituation has been shown to occur in \textit{Drosophila} due to interneuron plasticity \cite{das_plasticity_2011}, and the ability of \textit{Lepidoptera} species (moths) to pick out exquisite trails of pheromones in turbulent environments brimming with odoriferous sources indicates that the phenomenon is general. This background habituation task, related to figure-ground segregation problems, is a non-trivial statistical problem, but it is unclear how this problem is solved in the olfactory cortex of those various organisms. 
% Would need more, better examples of habituation. 

% Very short introduction
A broad variety of organisms are able to habituate to fluctuating olfactory background, thus enhancing their ability  and to recognize new odors that appear. The neural mechanisms enabling this habituation to fluctuating mixtures are not well understood but are of interest for the broad problem of figure-background segregation in time-varying signals. 

An olfactory network model implementing a simple habituation algorithm has been proposed recently \cite{shen_habituation_2020}. It achieves subtraction of the average background: post-habituation recognition of new odors improves when the background is constant, but performance is lost against fluctuating backgrounds. A more sophisticated neural mechanism is needed to achieve efficient habituation. %, to suppress fluctuations of the background as well as its average. 

The IBCM model of neural plasticity is a promising avenue to account for olfactory habituation. In this model, neurons become responsive specifically to certain components of the stimuli to which they are exposed. 
%Fluctuations of the stimuli actually drive neurons to respond to non-Gaussian projections of the input distribution.
The properties of IBCM neurons could therefore be leveraged to decompose a fluctuating olfactory background and detect the level of different background odors. Incidentally, synpaptic plasticity was found to be at the center of olfactory habituation in \textit{Drosophila melanogaster} \cite{das_plasticity_2011}. 

IBCM neurons cannot inhibit the olfactory input themselves, since their output is a scalar activity, not a vector that could be subtracted from the input mixture. We propose to couple IBCM neurons to inhibitory interneurons of the kind used by \cite{shen_habituation_2020}. The IBCM neurons will become specific to certain components of the olfactory background, and control their associated inhibitory neurons to suppress those components. New odors can still be detected because no IBCM neuron is yet specific to them. 

After reviewing the olfactory habituation network proposed by \cite{shen_habituation_2020}, we introduce the IBCM model of neuronal plasticity and our proposed inhibition network leveraging these neurons. We then extend existing analytical results on the IBCM model, which concerned finite sets of alternating input stimuli, to fluctuating linear mixtures of stimuli of increasing complexities: Gaussian binary mixture, general Gaussian mixtures, and weakly non-Gaussian mixtures. In each case, we also calculate the inhibition achieved by the proposed network and compare our analytical results to simulations. We also quantify the new odor detection performance in the weakly non-Gaussian case. We then study numerically strongly non-Gaussian background fluctuations (namely, log-normal fluctuations). 

There is still work to be done on the latter topic, and also to test the model against realistic concentration fluctuations given in \cite{celani_odor_2014} and with input odor vectors based on actual ORN activity data \cite{hallem_coding_2006}. 


% Again, too detailed: keep this for section on IBCM+inhibition network model
% On a slow time scale, each IBCM neuron becomes specific to a background component, and its associated inhibitory neuron learns that component by an averaging mechanism weighted by the IBCM neuron's activity. On a fast time scale (of odor fluctuations and neural firing), the IBCM neuron's activity controls the inhibitory neuron's level of inhibition (such that it is only acting when the appropriate odor component is present). Hence, after an habituation period, these pairs of neurons track the rapid fluctuations of the background and suppress them in real time. New odors can still be detected because no IBCM neuron is yet specific to them, so they are not subject to inhibition. 